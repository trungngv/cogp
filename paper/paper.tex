\documentclass{article} % For LaTeX2e
\usepackage[a4paper,margin=1.5in]{geometry}
\usepackage[mathlines]{lineno}
\usepackage{hyperref}
\usepackage{url}
\usepackage{amsmath}
\usepackage{graphicx}
\usepackage{bm}
\usepackage{hyperref}
\usepackage{natbib}
\usepackage{latexsym,amsbsy,amssymb,color,xspace, booktabs}

\include{macros}
\newcommand{\entropy}{\calH}

\newcommand{\sigmak}{\sigma_{k}}
\newcommand{\der}{\mathrm{d}}

\newcommand{\sigmanoise}{\sigma_{\text{n}}}
\newcommand{\sigmoid}{\text{sig}}
\newcommand{\Xu}{\mat{X}_u}
\newcommand{\bs}{\boldsymbol}
\newcommand{\xstar}{x_*}
\newcommand{\fstar}{f_*}
\newcommand{\ystar}{y_*}
\newcommand{\zstar}{z_*}
\newcommand{\vxstar}{\x_*}
\newcommand{\vfstar}{\f_*}
\newcommand{\Xuk}{\Xu^{(k)}}
\newcommand{\Kab}[2]{\mat{K}_{\vec{#1},\vec{#2}}}

% model specification
\newcommand{\BigMu}{\bs{\mu}}
\newcommand{\BigSigma}{\mat{\Sigma}}
\newcommand{\Mu}{\BigMu}
\newcommand{\Muk}{\bs{\mu}_k}
\newcommand{\Sigmak}{\mat{\Sigma}_k}
\newcommand{\hatMuk}{\bs{\hat \mu}_k}
\newcommand{\hatSigmak}{\mat{\hat \Sigma}_k}
\newcommand{\hatMukprime}{\bs{\hat \mu}_{k'}}
\newcommand{\hatSigmakprime}{\mat{\hat \Sigma}_{k'}}
\newcommand{\thetak}{\bs{\theta}_k}
\newcommand{\BigTheta}{\bs{\theta}}
\newcommand{\bsigma}{\bs{\sigma}}
\newcommand{\map}{\text{\sc{map}}}
\newcommand{\zmap}{\vz_\text{\sc{map}}}
\newcommand{\fuk}{\vec{g}_k}
\newcommand{\fu}{\vec{g}}
\newcommand{\kernelk}{k^{\BigTheta_k}}
\newcommand{\matUk}{\mat{U}_k}
\newcommand{\matXk}{\mat{X}_k}
\newcommand{\Lambdak}{\Lambda^{\BigTheta_k}}
\newcommand{\matK}{\mat{K}}
\newcommand{\matLambdak}{\mat{\Lambda}_k}

% variational inference
\newcommand{\Mufk}{\BigMu_k}
\newcommand{\Sigmafk}{\BigSigma_k}
\newcommand{\Kuk}{\matK^{(k)}_{\vu,\vu}}
\newcommand{\Psik}{\mat{\Psi}_k}
\newcommand{\Gammak}{\mat{\Gamma}_k}

\def\linenumberfont{\normalfont\small\sffamily}

\title{Multiple Output Regression}

\newcommand{\fix}{\marginpar{FIX}}
\newcommand{\new}{\marginpar{NEW}}

%\nipsfinalcopy % Uncomment for camera-ready version

\begin{document}
\maketitle
\linenumbers

\section{Model Description}
We describe a multiple-output regression model.
Suppose that we have inputs $\X = \{\vx_n\}_{n=1}^N$ and outputs $\vy = \{\vy_n\}_{n=1}^N$, with $\vx_n \in \calR^D$ and $\vy_n \in \calR^P$, i.e. the output is $P$-dimensional.
For ease of exposition, we assume that there are no missing values in any output dimension.
The most dominant approach in GP-based multiple output regression is to compose the different outputs as some \textit{linear} mixing (combination) of some basic processes (random functions).
Our model is also underpinned by this composition approach.
However, to really address the issue of scalability, this model is built upon sparse GPs from the ground up.


\subsection{Prior and Likelihood}
We use a simple composition:
\begin{align}
f_i(\vx) = w_i g(\vx) + h_i(\vx), \quad i = 1 \hdots P,
\end{align}
where each (latent) output is the sum of two functions: one scaled latent function that is shared by all outputs and one latent function unique to the output.
This allows the outputs to be correlated via the shared function $g(\vx)$ while still having their own independent processes.
Furthermore we assume that $g(\vx) \sim \GP(0, k^g(\vx,\vx'; \vectheta^g))$ and $h_i(\vx) \sim \GP(0, k_i(\vx,\vx'; \vectheta_i))$, and that $h_i$ are independent GPs with their own set of hyperparameters. \\

\noindent The auto-variance of the latent output using this composition is
\begin{align}
\nonumber
cov[f_i(\vx), f_i(\vx')] 
&= cov \left[ \left(w_i g(\vx) + h_i(\vx) \right) \left(w_{i} g(\vx') + h_i(\vx') \right) \right] \\
&= w_i^2 k^g(\vx, \vx') + k_i(\vx,\vx').
\end{align}
\\
\noindent The cross-covariance of the latent outputs is
\begin{align}
\nonumber
cov[f_i(\vx), f_{i'}(\vx')] 
&= cov \left[ \left(w_i g(\vx) + h_i(\vx) \right) \left(w_{i'} g(\vx') + h_{i'}(\vx') \right) \right] \\
&= w_i w_{i'} k^g(\vx, \vx').
\end{align} 

\noindent It can be seen that the induced covariance function of $f_i(\vx)$ is very similar to that in the MTGP model.
Specifically, in MTGP the task-correlation matrix is a free-form positive definite matrix $B$ whereas here it is the matrix $\vw \vw^T$ where $\vw = [w_1, \hdots, w_P]^T$.
However this model is more flexible compared to MTGP because each output is also influenced by its own independent process $h_i(\vx)$.
Note that we can also use more than one shared function $g(\vx)$, in which case it becomes the semiparametric latent force model (SLFM), again with the addition of independent process in each output dimension. \\

\noindent 
\textbf{Augmented sparse GPs}
Toward scalable modeling, we replace standard GPs with sparse GPs augmented with \textit{different} set of inducing inputs.
This adds much flexibility to the model as the roles of $g(\vx)$ and $h_i(\vx)$ can be quite different, so it is necessary that each has its own inducing inputs.
Furthermore, $g(\vx)$ can be seen as a \textit{global} function operating on the entire input space (of all output dimensions), while each $h_i(\vx)$ operates only on the inputs of the $i$-th output, which can be smaller than that of $g(\vx)$.
If more than one $g(\vx)$ is used, each $g(\vx)$ may capture a different pattern so they should also have separate inducing inputs. \\

\noindent
\textbf{Prior}
\newcommand{\Zg}{\mat{Z}^g}
\newcommand{\Zi}{\mat{Z}_i}
Let $\vg = g(\X), \vh_i = h_i(\X), \vh = \{\vh_i \}_{i=1}^P$. Let $\Zg$ and $\mat{Z}^h = \{\Zi \}_{i=1}^P$ be the set of inducing inputs for $g(\vx)$ and $h_i(\vx)$; the corresponding inducing points $\vu^g$ and $\vu^h = \{\vu_i \}_{i=1}^P$.
The prior of the augmented model is given by:
\begin{align}
p(\vg | \vu^g) &= \Normal(\vg; \BigMu^g, \tilde{\K}^g )\\
p(\vu^g) &= \Normal(\vu^g; \vec{0}, k^g(\Zg, \Zg)) \\
p(\vh | \vu^h) &= \prod_{i=1}^P \Normal(\vh_i; \BigMu_i, \tilde{\K}_i)\\
p(\vu^h) &= \prod_{i=1}^P \Normal(\vu_i; \vec{0}, k_i(\Zi, \Zi)),
\end{align}
where $\BigMu^g = k^g(\X,\Zg)k(\Zg,\Zg)^{-1}\vu^g$ and $\tilde{\K}^g = k^g(\X,\X) - k(\X,\Zg)k(\Zg,\Zg)^{-1}k(\Zg,\X)$ and $\BigMu_i, \tilde{\K}_i$ are similarly defined.

\noindent
\textbf{Likelihood}
The likelihood as usual follows the standard iid Gaussian likelihood,
\begin{align}
p(\vy | \vg, \vh ) = \prod_{i=1}^P p( \vy_i ; \vg, \vh_i) = \prod_{i=1}^P \Normal( \vy_i ; \vg + \vh_i, \beta_i^{-1} \I).
\end{align}

\subsection{Variational Inference in Sparse GPs Revisited}
In this section we review inference in sparse GPs (as presented in Titsias).
The posterior in an augmented model is $p(\vf, \vu | \vy) = p(\vf | \vu, \vy) p(\vu | \vy)$.
The key property of such augmented GPs  is the notation of \textit{sufficient statistics}: given the inducing points $\vu$, the latent values $\vf$ are independent with any other set of latent values (e.g. the test set).
In the optimal setting when $\vu$ is the sufficient statistics of $\vf$, it should hold that $p(\vf | \vu, \vy) = p(\vf | \vu)$ as $\vy$ is only the noisy version of $\vf$.
This leads to choosing a variational approximation of the posterior which factorizes as $q(\vf, \vu | \vy) = p(\vf | \vu) q(\vu | \vy)$.
Since the conditional $p(\vf | \vu)$ is known, variational inference becomes learning an optimal posterior $q(\vu | \vy)$ only. \\

\noindent Also as a consequence of sufficient statistics, the approximate prediction for test targets $\vfstar$ is
\begin{align}
\nonumber
p(\vfstar | \vy, \X_*) &= \int p(\vfstar | \vf, \vu, \X_*) q(\vf, \vu | \vy) \der \vf \der \vu \\
\nonumber
&= \int p(\vfstar | \vu) q(\vu| \vy) p(\vf | \vu) \der \vf \der \vu \\
\label{eq:sorprediction}
&= \int p(\vfstar | \vu) q(\vu| \vy) \der \vu
\end{align}
which is the same as the prediction of a SoR model (of course with a different posterior in $q(\vu | \vy)$). 

\subsection{Variational Inference}
\newcommand{\ug}{\vu^g}
\newcommand{\uh}{\vu^h}
Our goal of inference is to find the posterior $p(\vg, \vh, \ug, \uh | \vy)$. 
Following the previous discussion, we assume a variational distribution which factorizes as:
\begin{align}
\nonumber
q(\vg, \vh, \ug, \uh | \vy)
 &= p(\vg | \ug) q(\ug | \vy) \prod_{i=1}^P p(\vh_i | \vu_i) q(\vu_i | \vy) \\
 &=  q(\ug, \uh) p(\vg | \ug) \prod_{i=1}^P p(\vh_i | \vu_i)
\end{align}
Since the conditionals $p(\vg | \ug)$ and $p(\vh_i | \vu_i)$ are given, we need only to find the optimum $q(\ug, \uh) = q(\ug | \vy) \prod_{i=1}^P q(\vu_i | \vy)$.
Let $q(\ug | \vy) = \Normal(\ug; \vm^g, \S^g)$ and $q(\vu_i | \vy) = \Normal(\vu_i; \vm_i, \S_i)$. \\

\noindent
To find the optimum $q(\ug, \uh)$ we optimize the evidence lowerbound of the log marginal,
\begin{align}
\nonumber
\log p(\vy) \ge& \int q(\ug, \uh) \log \frac{p(\vy | \ug, \uh) p(\ug, \uh)}{q(\ug, \uh)} \der \ug \der \uh \\
\nonumber
=& \int q(\ug, \uh) \log p(\vy | \ug, \uh)  \der \ug \der \uh 
+ \int q(\ug, \uh) \log \frac{p(\ug, \uh)}{q(\ug, \uh)} \der \ug \der \uh \\
\label{eq:elbo}
=& \int q(\ug, \uh) \log p(\vy | \ug, \uh)  \der \ug \der \uh 
- \left( KL[q(\ug) || p(\ug)] + \sum_{i=1}^P KL[q(\vu_i) || p(\vu_i)] \right),
\end{align}
where the last equality occurs because both of $q(\ug, \uh)$ and $p(\ug, \uh)$ fully factorize.
Since $q(\ug), q(\vu_i), p(\ug), p(\vu_i)$ are all multivariate Gaussian distribution, the KL divergences are analytically tractable and require $\calO(M^3)$ computation, where $M$ is the largest number of inducing inputs (recall that $g(\vx)$ and $h_i(\vx)$ use separate set of inducing inputs). \\

\noindent To compute the above equation we first focus on the term:
\newcommand{\llangle}{\left\langle}
\newcommand{\rrangle}{\right\rangle}
\begin{align}
\nonumber
\log p(\vy | \ug, \uh)
 &= \log \llangle p(\vy | \vg, \vh) \rrangle_{p(\vg,\vh | \ug, \uh)} \\
 \nonumber
&\ge \llangle \log p(\vy | \vg, \vh) \rrangle_{p(\vg,\vh | \ug, \uh)} \quad &\text{(Jensen's inequality)} \\
\nonumber
&=  \llangle \sum_{i=1}^P \sum_{n=1}^N \log p(y_{in} | g_n, h_{in}) \rrangle_{p(\vg,\vh | \ug, \uh)}  \quad &\text{(factorized likelihood)} \\
&= \sum_{i=1}^P \sum_{n=1}^N \llangle \log p(y_{in} | g_n, h_{in}) \rrangle_{p(\vg,\vh_i | \ug, \vu_i)} \quad &\text{(linear operation)}
\end{align}
Observe that this is very similar to the SVI for standard GP in Hensman et al. 
In this multiple-output model, the outputs factorize given $\vg$ and $\vh$ hence the lowerbound can be seen as sum of the lowerbounds in the single-output setting.

\noindent Each individual term $l_{in} = \llangle \log p(y_{in} | g_n, h_{in}) \rrangle_{p(\vg,\vh_i | \ug, \vu_i)}$ can be computed as follows:
\begin{align}
\nonumber
l_{in} &= \int \log p(y_{in} | g_n, h_{in}) p(\vg | \ug) p(\vh_i | \vu_i) \der \vg \der \vh_i \\
\nonumber
&= \int \log \Normal(y_{in} ; w_i g_n + h_{in}, \beta_i^{-1}) 
\Normal(g_n ; \mu^g_n, \tilde{k}^g_{nn})
\Normal(h_{in} ; \mu_{in}, (\tilde{k}_{inn}) \der g_n \der h_{in} \\
\nonumber
&= -\frac{1}{2} \log 2 \pi \beta_i^{-1} - \frac{1}{2} \int (y_{in} - w_i g_n - h_{in}) \beta (y_{in} - w_i g_n - h_{in})
\Normal(g_n ; \mu^g_n, \tilde{k}^g_{nn})
\Normal(h_{in} ; \mu_{in}, \tilde{k}_{inn} \der g_n \der h_{in} \\
\nonumber
&= -\frac{1}{2} \log 2 \pi \beta_i^{-1} - \frac{1}{2} \int \left[(y_{in} - h_{in} - w_i \mu^g_n) \beta (y_{in} - h_{in} - w_i \mu^g_n) + w_i^2 \beta \tilde{k}^g_{nn} \right] 
\Normal(h_{in} ; \mu_{in}, \tilde{k}_{inn} \der h_{in} \\
\nonumber
&= -\frac{1}{2} \log 2 \pi \beta_i^{-1} - \frac{1}{2} w_i^2 \beta \tilde{k}^g_{nn}
- \frac{1}{2} \beta \tilde{k}_{inn} - \frac{1}{2} (y_{in} - w_i \mu^g_n - \mu_{in}) \beta (y_{in} - w_i \mu^g_n - \mu_{in}) \\
&= \log \Normal(y_{in}; w_i \mu^g_n + \mu_{in}, \beta_i^{-1})  - \frac{1}{2} \beta_i (w_i^2 \tilde{k}^g_{nn} + \tilde{k}_{inn}).
\end{align}
Notice that the above expression is very similar to the single-output case.
Hence our derivation here can be seen as the \textit{generalization} of SVI from the standard GP regression to multiple-output regression.

\noindent Substituting $l_{in}$ into equation \ref{eq:elbo} and carrying similar integration we get,
\begin{align}
\log p(\vy)
\ge& \sum_{i=1}^P \sum_{n=1}^N
 \left( \log \Normal(y_{in}; \tilde{\mu}_{in}, \beta^{-1})
         - \frac{1}{2} \beta_i (w_i^2 \tilde{k}^g_{nn} + \tilde{k}_{inn})
         - \frac{1}{2} \trace \left( w_i^2 \S^g \mat{\Lambda}^g_n + \S_i \mat{\Lambda}_{in} \right)
\right) \\
&- \left( KL[q(\ug) || p(\ug)] + \sum_{i=1}^P KL[q(\vu_i) || p(\vu_i)] \right),
\end{align}
where 
\begin{align}
\tilde{\mu}_{in}
&= w_i k^g(\vx_n, \Zg)k^g(\Zg,\Zg)^{-1}\vm^g + k_i(\vx_n, \Zi)k_i(\Zi,\Zi)^{-1}\vm_i \\
\mat{\Lambda}^g_n
&= \beta k^g(\Zg,\Zg)^{-1} k^g(\Zg, \vx_n) k^g(\vx_n, \Zg) k^g(\Zg,\Zg)^{-1} \\
\mat{\Lambda}_{in}
&= \beta k_i(\Zi,\Zi)^{-1} k_i(\Zi, \vx_n) k_i(\vx_n, \Zg) k_i(\Zg,\Zg)^{-1}.
\end{align}

\noindent Again, this clearly shows that this model generalizes the standard GP regression. This can be verified by setting $P = 1$, $w_i = 1$ and $h_i(\vx) = 0$ to recover the bound in Hensman et al.
Due to the decomposition of this bound, we can use stochastic gradient descent to learn the variational parameters.

\subsection{Prediction}
The predictive distribution of the $i$-th output for a test input $\vx_*$ is 
\begin{align}
p(\fstar | \vy, \vx_*) = \int \Normal(\fstar; w_i g_* + h_{i*}, 0) p(g_* | \vy, \vx_*) p(h_{i*} | \vy, \vx_*) \der g_* \der h_{i*},
\end{align}
where $p(g_* | \vy, \vx_*) = \Normal(g_*; \mu^g_*, s^g_*)$ and $p(h_{i*} | \vy, \vx_*) = p(h_{i*}; \mu_{i*}, s_{i*})$ are the predictive distributions of the sparse GPs as given in \ref{eq:sorprediction}.
Therefore we have:
\begin{align}
p(\fstar | \vy, \vx_*) = \Normal(\fstar; w_i \mu^g_* + \mu_{i*}, s^g_* + s_{i*}). 
\end{align}

%\begin{linenomath}
%\begin{align}
%\sum_{i,h} \lambda_i \lambda_h cov [f(\vx, \vx')]
%&= \sum_{i,h}   \sum_{j,j'} \lambda_i g(\vx_i - \vs_j) k(\vs_j, \vs_{j'}) \lambda_h g(\vx_h - \vs_j') \\
%\end{align}
%\end{linenomath}
  
\end{document}
